% Use only LaTeX2e, calling the article.cls class and 12-point type.
\documentclass[12pt]{article}
%\usepackage{times}
\usepackage{booktabs}
\usepackage{lineno}% linenumbering
%\usepackage[onehalfspacing]{setspace} % onehalf spacing
\usepackage{graphicx}% add graphics
\usepackage[utf8]{inputenc}% encoding
\usepackage{gensymb}% symbols
\usepackage{textcomp}% symbols
\usepackage[document]{ragged2e}% left justify
\setlength{\RaggedRightParindent}{\parindent}%indent
\usepackage[labelsep=period,labelfont=bf,singlelinecheck=false,figurename=Fig.]{caption}% caption format

% Hyperlinking
\usepackage{url}% for better urls
\renewcommand{\UrlFont}{\small\ttfamily}% make urls smaller
\PassOptionsToPackage{hyphens,spaces,obeyspaces}{url}\usepackage{hyperref}% to avoid clashes between url and hyperref
\hypersetup{breaklinks=true,bookmarks=true,colorlinks=true,urlcolor=black,linkcolor=black,citecolor=black,pdfauthor={Rupert A.\ Collins},pdftitle={delimtools: an R package to simplify species delimitation analyses}}
\renewcommand*{\figureautorefname}{Fig.}

% Bibliography customisation 
\usepackage[apaciteclassic]{apacite}%,nodoi
\usepackage{natbib}
\bibliographystyle{apacite}


% The following parameters seem to provide a reasonable page setup.
\topmargin 0.0cm
\oddsidemargin 0.2cm
\textwidth 16cm 
\textheight 21cm
\footskip 1.0cm

% new cprime commands
%\newcommand*{\mins}{$^{\prime}$}
%\newcommand*{\secs}{$^{\prime\prime}$}

%The next command sets up an environment for the abstract to your paper.
\newenvironment{sciabstract}{%
\begin{quote} \bf}
{\end{quote}}

% Include your paper's title here
\title{\textsl{delimtools}:\ an R package to combine and simplify single-locus species-discovery analyses} 

\author
{Pedro Senna Bittencourt$^{1\ast}$, Tomas Hrbek$^{1}$, Rupert A.\ Collins$^{2}$\\ \\
\\
\small{$^{1}$Laboratório de Evolução e Genética Animal, Departamento de Biologia,}\\
\small{Universidade Federal do Amazonas, Manaus, AM, Brazil}\\
\small{$^{2}$Department of Life Sciences, Natural History Museum,}\\
\small{Cromwell Road, London SW5 7BD, UK}\\
\\
\normalsize{$^\ast$To whom correspondence should be addressed; E-mail:\ pedro\_bittencourt@yahoo.com.br.}
}
% Include the date command, but leave its argument blank.
\date{}

%%%%%%%%%%%%%%%%% END OF PREAMBLE %%%%%%%%%%%%%%%%
%%%%%%%%%%%%%%%%% END OF PREAMBLE %%%%%%%%%%%%%%%%
%%%%%%%%%%%%%%%%% END OF PREAMBLE %%%%%%%%%%%%%%%%

\begin{document} 

% Double-space the manuscript.
\baselineskip24pt

% Make the title.
\maketitle 
\newpage
\linenumbers
% Place your abstract within the special {sciabstract} environment.
\begin{sciabstract}
\section*{Abstract}
Summary of delimtools.%

\bigskip
Keywords: species discovery, taxonomy, DNA barcoding.
\end{sciabstract}

\newpage
\section*{Introduction}

Explain species delimitation and taxonomy.%

Sumarise currently available tools.%

Describe the need for delimtools.%

Aims of the paper.%

%%%%%%%%%%%%%%%%%%%%%%%%%%%%%%%%%%%%%%%%%%%%%%%%%%%%%%%%%%%%%%%%%%
%%%%%%%%%%%%%%%%%%%%%%%%%%%%%%%%%%%%%%%%%%%%%%%%%%%%%%%%%%%%%%%%%%

\section*{Materials \& methods}

\subsection*{Software description}

How does \textsl{delimtools} work.%

\subsection*{Example}

Example of species delimitation using \emph{Geophagus} data from \citet{Ximenes2021}.%

%%%%%%%%%%%%%%%%%%%%%%%%%%%%%%%%%%%%%%%%%%%%%%%%%%%%%%%%%%%%%%%%%%
%%%%%%%%%%%%%%%%%%%%%%%%%%%%%%%%%%%%%%%%%%%%%%%%%%%%%%%%%%%%%%%%%%

\section*{Results}

\subsection*{\emph{Geophagus} delimitation}

%%%%%%%%%%%%%%%%%%%%%%%%%%%%%%%%%%%%%%%%%%%%%%%%%%%%%%%%%%
%%%%%%%%%%%%%%%%%%%%%%%%%%%%%%%%%%%%%%%%%%%%%%%%%%%%%%%%%%

\section*{Discussion}

Summary of work

\subsection*{Acknowledgements}

We thank xxx.%

%%%%%%%%%%%%%%%%%%%%%%%%%%%%%%%%%%%%%%%%%%%%%%%%%%%%%%%%%%%%%%
%%%%%%%%%%%%%%%%%%%%%%%%%%%%%%%%%%%%%%%%%%%%%%%%%%%%%%%%%%%%%%

\bibliography{delimtools}

\newpage
\subsection*{Tables \& Figures}

\begin{table}[htbp]
\scriptsize
\caption{fig legend}
\begin{tabular}{ll}
\toprule
Delimitation & No.\ spp.\\
\midrule
A & 1 \\
B & 2 \\
C & 3 \\
\bottomrule
\end{tabular}
\label{tab:xxx}
\end{table}


\newpage

%\noindent \textbf{Figure 1}. 
\begin{figure}[!htbp]
\caption{fig legend.}
\begin{center}
%\includegraphics[width=0.7\textwidth]{file.tif}
\end{center}
\label{fig:xxx}
\end{figure}



\newpage
\section*{Supplementary materials}

\noindent \textbf{Table S1}. Legend.%
\bigskip

\noindent \textbf{Fig.\ S1}. Legend.%
\bigskip

\newpage 

\clearpage
\end{document}




















